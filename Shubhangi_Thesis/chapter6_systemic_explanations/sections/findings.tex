In the workshops, diverse local publics partially explained components of AI systems such as: (1) spatial, social, political, and historical characteristics of the places that are subject to predictions, (2) institutions, cultures, and contexts that algorithms engage with, and (3) lived experiences of public safety and related resource allocations including care, financial, health, food, and shelter resources. These explanations revealed novel features of the broad and complex AI system, prompting critical inquiry into how these features could impact society. 

\subsection{Experience of safety in diverse spaces} 

Participants explained the on-the-ground reality of spaces that are subject to predictions. In workshop 2, the group was talking about the places they think are most in need of public safety aid. One of the participants talked about a specific neighborhood where he goes to work every day. He explains: 

\begin{quote}
    This is where our office is and it is in Mercy Care which is like a free clinic so 15, 16 thousand mostly unhoused uninsured patients a year. And there is this bridge on Hillyard street and there are people who sleep under that bridge. So I was thinking of the vulnerabilities of the people who live around it so they can access medical services but don't have housing. They are exposed to all kinds of things.. It's also gentrifying, so they are getting displaced further away from where the services are. We have pretty much criminalized homelessness, so they are maybe at risk of police interactions or something. [W2]
\end{quote}

As he described this area, he highlighted the importance of considering whose safety are predictive tools prioritizing. With a better understanding of the realities of spaces subject to predictions, we can assess how the deployment of more police forces may affect the diversity of people that inhabit the spaces. He also talked about the socio-economic characteristics of the neighborhood— rising gentrification and increasing distance from health care services and how that affects the safety of people living there. Lastly, he explains the social and political tendency to criminalize homelessness. These tendencies and corresponding policies dictate what counts as a ‘crime’ and who is detained by the police deployed by a predictive tool. In another workshop (W3), participants explained the effects of the use of a predictive tool in a specific neighborhood. One of the explains: 

\begin{quote}
    I can tell you that a predictive factor in Mechanicsville will not be welcome. No one wants to be overpoliced. Mechanicsville, East Atlanta, where gentrification is happening, people are being pushed out of their communities and then they are being heavily policed. That doesn't feel safe, that feels unsafe. Something to think about is that how does it land on people who have been disinvested in and marginalized. These tools don't tell the whole story. They say they want to address safety issues in this bubble, that is law enforcement and not system changes. [W3]
\end{quote}


The participant directly explained how the use of the tool will affect specific communities and if and how it may work to reinforce ongoing injustices. She provides a clear explanation of the oppressive contexts in which the tools will be deployed and calls for addressing issues not by law enforcement that will further marginalize communities but by systemic change. 

\subsection{Norms and cultures in space }

In workshop 1, a participant that works for a police reform group describes her interactions with reports for criminal trespassing in upper-class wealthy neighborhoods of Buckhead. She explains: 

\begin{quote}
    What we see on a daily basis, areas like Buckhead, they give us the most referrals for criminal trespassing, and it could just be an unhoused person sitting in the park in the middle of the day. They just don't wanna see them. But on the south side, the more impoverished communities, we don't get calls like that for unhoused people because they are friends and family, they don't look at them like criminals and they support them. [W1]
\end{quote}

Overreporting often invites more police presence resulting in unwanted violent encounters between police forces and marginalized communities. Sometimes, as another participant in workshop 3 explains, community members are encouraged to call 911 repeatedly so more police forces are deployed in their areas: 

\begin{quote}
    And the police in neighborhoods are not shy of telling people you need to call 911 to get that data point because that helps direct the zone commanders to localize resources to those areas. So, some people, maybe not excessively, but some folks are .. they would become hotspots where it has become the social norm to call 911 in comparison to other areas where thats not something..[W3]
\end{quote}

The participant continues to explain how some neighborhoods have special committees designed specifically to address crime by requesting more police presence and surveillance cameras or even self-policing their neighborhoods by registering retired police officers as volunteers. On the other hand, some neighborhoods prevent police encounters as much as possible. Such norms and cultures in different neighborhoods are reflected in place-based data that inform the predictions made by algorithmic tools. 

\subsection{Data Contexts }

One of the participants in Workshop 3 leads a violence prevention program affiliated with one of the largest hospitals in the city of Atlanta. As part of the program, they analyze data related to firearm injuries, while working with community partners to reduce the risk of recurrent injury. She describes the data her team collects and explains: 

\begin{quote}
    And if we are talking about violent crime, one of the things that I wanna say is that 80 \% of our cases are not reported to law enforcement.. So there is huge discrepancy in what we see at the hospital and what we know ..what APD (Atlanta Police Department) or Dekalb county’s data reflects.. [W3]
\end{quote}

Through her work, she has access to data that explains the limits of existing crime-related data. Many people who come to her hospital to seek care do not officially report the crimes that they become victims of with the police department. Their reasons may vary but it foregrounds the sheer amount of data that is missing from the training data. 

\subsection{Explaining spatial characteristics} 

`Five points' was an area repeatedly identified as a crime hotspot across workshops. In workshop 2, which brought together a group of urban planners with an understanding of urban space and its effects, one participant explained why five points may be seeing a high crime rate: 

\begin{quote}
    I have seen people get ticketed for jay walking at five points. I am not sure anywhere would ever get ticketed for that anywhere else and it is like they were pretty specifically targeted. It is like they are trying to keep people out of certain spaces. And to catch a bus, you have to jay walk, there is no crosswalk.
\end{quote}

As he described targeting people for petty crimes in five points, he also highlighted how the spatial features of the area make it easier for police to ticket people over crimes like jaywalking. Later in the workshop, a participant described another such intersection where the spatial features promote an increased police presence: 

\begin{quote}
    Another thing that is very unique right here [highly policed area] is that there is cross-jurisdictional cooperation between 13 different police departments. There are 13 different kinds of police that can stop you, arrest you.. It’s a residual policy from the Olympics. It is a huge reason why a lot of these spaces are so policed because they are gigantic mechanisms for moving people in and out for events and protecting their cars essentially when they are inside.
\end{quote}

He explained how historical events, such as the 1996 Olympics held in Atlanta, informed the design and segregation of spaces in lasting ways. Such segregation, that were designed for alternate purposes and contexts, continues to affect the people who live in those spaces today. 

\subsection{Explaining policing institutions} 

Participants in Workshop 1 worked as part of a police reform group. They offered care and support services to people experiencing extreme poverty, problematic substance use, or mental health concerns to reduce arrest rates. They work in collaboration with police departments who may sometimes divert 911 calls to their organization. Participants in this organization were driven by the cause they support, seeking police reform. They were knowledgeable about policing works and its many shortcomings and were able to explain the institution of policing in relevance to the predictive tool. One participant explains: 

\begin{quote}
    I will say that it actually started as a way of slave control. Then it became something that helps the white class in the reconstruction era to control free black people. And so it evolved from that. Until there is true reform in how these traditions of policing, then you can't bring something like this in the picture. [W1]
\end{quote}
They explain the historical origins of policing that are still prevalent in how police operate today. With that explanation, they help us understand what is needed to promote public safety in ways that challenge the historical underpinning of the institution and policing and make way for more just practices. They also explain the current state of prisons, specifically in Georgia: 
\begin{quote}
    Georgia has the most private for-profit prisons of any state in the entire country. And there are a lot of large corporations that use that free labor. [W1] 
\end{quote}
This explanation helps us understand and question the underlying incentives of predictive tools. Such a financial incentive can push large corporations to influence the workings of predictive systems in ways that drive more people into prisons for increased profits.  

\subsection{Explaining desirable futures} 

In workshop 4, a participant explained how civic organizations can work with the city to offer care when they observe a pattern of recurring crime: 

\begin{quote}
    Downtown in Woodridge Park, which is right here [shows on map], has a high homeless population, they tend to stay in the park all day, there are a lot of businesses in the area who hand out food. Woodridge park recognized that this is a different kind of ‘crime’ so the park worked with one of the agencies downtown to have a resource officer staffed as an employee right outside the park instead of a police officer who works with the people in the area to adjust their needs and to get them out of the streets and into support services, so that is an example of recognizing a pattern of a certain kind of crime in an area and addressing it through a tailored approach.
\end{quote}

Such real-world explanations of how we can address certain kinds of patterns and predictions with an approach that centers the needs of the marginalized can help us understand the kinds of impact predictive tools could aim to have. Such explanations can guide the redesign of protocols following a prediction to involve the use of civic resources, eventually leading to systemic changes. 

These instances described above demonstrate how diverse publics may be well-equipped to explain parts of an AI systems that they engage with. 