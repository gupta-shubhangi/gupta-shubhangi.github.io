This chapter aims to investigate if and how local publics can partially explain elements of AI systems in the capacities in which they interact with these systems. To do so, I start with identifying and pulling out instances of partial explaining and questioning from the workshops to form a repository. Next, I write memos \cite{bernard2017research} reflecting on if and how this explanation can help us better understand AI systems as socio-technical assemblages alongside novel questions that it opens up. 

This was followed by a reflexive thematic analysis \cite{braun2019reflecting} where I categorized partial explanations offered by publics into elements of AI systems that they explained such as—spatial, social, relational, historical, policy, lived, etc. Through further refinement of these themes I identified the categories of explanations that organize the findings section below. These categories are not exhaustive or mutually exclusive but serve as an example of the variety of partial explanations users can provide about socio-technical AI systems. 


