In this section, I report on the explanation contexts that emerged as I mediated processes of knowing and explaining predictive systems. Out of the five workshops I conducted, I briefly describe partial contexts of Workshop, 1, 4, and 5. These workshops serve as good exemplary cases for this work. 

\subsection{Workshop with police reform group (W1)}

W1 invited twelve members of a police reform group that offers supportive services to those experiencing extreme poverty, problematic substance use, or mental health concerns, with the goal of reducing their arrest and incarceration rates. At the beginning of the workshop, participants expressed their feelings toward predictive policing tools calling them \emph{“really terrifying”} and \emph{“scary”}. Their fear was driven by \emph{“rights issues”, “reading of cases of people being exonerated from convictions with AI-based evidence”, AI being “rolled out so fast”, “need for compassion in policing that AI lacks”, ability of AI to “decide the outcome of someone’s life and future”}. These feelings of concern motivated their presence and desire to know more about the workings and effects of predictive policing tools. Their collective skepticism towards predictive tools including \emph{“facial recognition"}  or  \emph{“ads”} drove their fear of place-based predictive policing. 

Owing to their position in and relationship with the institution of policing, they were highly skeptical of technology usage in policing. They firmly believed that ‘crime’, as defined by the law and police departments is unjust. As such, they inferred what crime would mean for predictive systems. They said:
\begin{quote}
    “[Our] definition of crime is very different than what AI systems refer to as crime. So, like drugs are not crime to us. So, this [predictive system] is a lot about like tracking drugs. But we think drugs are a public health issue and not a safety issue.” 
\end{quote}
Concerned that police may categorize public health issues as crime, they asked what definition of crime is the predictive tool using. They questioned the motives of predictive systems: \emph{“If you are going to criminalize these public health issues then it may be more beneficial to predict where there will be these public health issues”}. The intentions of the tool— to arrest people or help them— dictate the predictions a tool makes and the policing actions that follow the predictions. 

Drawing on their experiences working with and living in diverse communities, they discuss the disparity in neighborhoods. The relationship of the neighborhood residents with each other and their collective relationship with the police would directly affect the workings of predictive tools. They explain: 
\begin{quote}
    “What we see on a daily basis, areas like Buckhead, they give us the most referrals for criminal trespassing, and it could just be an unhoused person sitting in the park in the middle of the day. They just don't wanna see them. But on the south side, the more impoverished communities, we don't get calls like that for unhoused people because they are friends and family, they don't look at them like criminals and they support them.  My mom calls the person on 11 and 23 on Cambleton road, she calls him uncle. I know his face and can point him out. If you are familiar with them, you wanna see them living. You just want them to get help.” 
\end{quote}
Neighborhoods that consider outsiders to be 'threats' are more likely to report them to the police. Another participant builds on this and suggests: \emph{“If you go off of 911 calls you have to see who calls 911. Because not everyone does. So the data is already going to be skewed.”} They present a need to inquire how the data that is fed into predictive tools is collected and whose perspectives does it capture. They fear that disparate reporting patterns, if not acknowledged and addressed, can reinforce harmful policing practices. 

The group emphasizes that it is essential to consider not just the data being put in and the prediction that is the output, but what we do with the predictions: \emph{ “Are they gonna start going out and arresting people or are they gonna use the fourth amendment and protect our rights once that data is gathered. I think it is about controlling the police in a way that protects our rights.”} This opens another explanation need for learning about how police and other stakeholders act on the tool's  prediction. They also discuss what this action means for community trust in police and holding police accountable. A \emph{“disconnect between real events and AI predicting events will add to the distrust in a world where there is already a lot of distrust... how do we prevent them from terrorizing certain areas and hold them accountable for where they decide to go?” } Their worries and questions are grounded in the current state of the institution of policing. They say: \emph{“For predictive policing to have any effect of public safety, policing itself would have to have an effect on public safety...  ” } They ultimately ask about how this specific tool betters or worsens the current state of the institution of policing. 

\subsection{Workshop with civic funding agency (W4)}

W4 brought in 11 members of an organization that leads, manages, and funds local, community-centered projects across Georgia. They considered the workshop \emph{‘timely’}. Unlike Workshop 1 described above, most of the people in this group were \emph{‘cautiously optimistic’} and wanted to understand how they could make the most of predictive technologies while managing their harms. They hoped to deploy technologies to reduce crime and increase economic opportunities in the American South.  

When discussing predictive policing, participants speculated that the tool may drive police officers to areas with petty crime to increase their arrest count. An increase in arrest count may help demonstrate the effectiveness of the tool for wider adoption in the country. Given their understanding of the economic growth needed to fund projects, they expected a report of higher crime and arrest rates to be a financial motivator. However, they questioned whether an increase in arrests would truly promote public safety. This led to them asking:
\begin{quote}
    “I have a question about.. is that [arrest count] the metric police are using to say that they are efficient? I know historically they have used such a metric but recently there have been studies on how that is not the right metric and how arresting people for petty crimes is not the right metric. So, I wonder if police departments have started to shift their metrics that show they are efficient. If they have changed, then this tool may be able to help.” 
\end{quote}

As such, they inquired about the methods used to assess the effectiveness of predictive practices across police departments. They hoped that a more holistic method than 'increase in arrest count', could help prove useful to measure the impact of the tool. However, they stated \emph{“How can you know, all these police departments are autonomous in their own ways so who knows what their objectives are?” } They discuss the challenges of opaque and decentralized police departments that may measure and report their arrest counts or safety impacts differently. 

They draw parallels between crime data or 911 data with 311 data about non-emergent civic issues, such as infrastructure repair. From their work, they are aware that over-reporting of 311-related issues in certain areas draws economic investment there over other places. They ask how disparate reporting patterns may affect crime predictions.

During the workshop, participants marked high crime spots on the map based on their own perceptions of space. They revisit those markings and say \emph{ " we have so many circles here, we have little five points and waterbouys down here ".} They call the marked crimes in those areas \emph{“very minor and visible”} such as \emph{“drug use, which is not a crime that is actively hurting anyone”}. They were trying to make a point about how a quantitative measure of more may not always mean decreased safety for the public. They inquired about how the predictive tool measures the seriousness of crimes. 

The conversation about drug use and crime led a participant to ask \emph{“We have talked about public safety a lot. Have you looked at cities that have started to decriminalize certain things? Some of it is policy right? Historically some crimes should not be classified as crimes and should not be fed into the systems. How do these systems change with ordinances and laws? Like Houston, they decriminalized marijuana and one of their elected officials ran on the fact that arrests went down.” } They asked if such policy changes resulted in data cleaning or not.

In initial introductions before the start of the workshop, the director of the group said that according to her, poverty is the main cause of crime. This was reinforced multiple times in the workshops and led to participants asking \emph{“Are there efforts to predict why people commit crime instead of where? Because then maybe people need food in a certain neighborhood.” “Are we getting people the right services and what do those services look like?” } They asked about how predictive tools could be used to provide care services to those in need instead of arresting them using the force of police. They speculated possible responses together: 
\begin{quote}
    “Crime is too big of a genre .. to think about how we can do this.. There are different types of crimes that may need different responses, maybe we can see the different types of crimes that are happening and if there are pockets of different crimes in different places and deploy different kinds of resources...[another person continues] doesn't that call for partnerships, that's when the community improvement district steps up to work with the department of public safety, that works with the civic non-profit organization, etc coming together, to address a place-based problem that is globalized to a condition.” 
\end{quote}
Through these partnerships between different civic entities, we can begin to address place-based public safety concerns. They wondered how the predictive tool affects such partnerships currently as well as the potential it has in doing so in the future. 

\subsection{Workshop with educators (W5)}

W5 invited a group of 8 teachers who came together as a non-profit aimed at forming meaningful relationships and collaborations between educators. Participants taught subjects such as the theory of knowledge, geography, and high school science in different schools spread out in the city of Atlanta. Many of them were participating to better equip themselves with information on civic AI systems in a world where they are suddenly being \emph{“bombarded by AI”}. Additionally, they sought skills that would allow them to train young minds to be more critical as they grow up in the world of predictive systems.  

The workshop started with a discussion about the goals of place-based predictive policing, i.e. efficient deployment of police forces. Participants talked about their desire for a \emph{“layered response”}. They compared policing practices in the United States with other countries and explained \emph{“here we only have one type of cop with a gun—they are reactive, in other countries, there are cops walking around whose response is not gun”}. They also compared current policing mechanisms with historical accounts: \emph{“I remember when we used to have sub stations... It would be nice to have police on foot who are walking around and getting to know people. Like the oldies. There are programs where police are able to get affordable housing in neighborhoods that have more crime. They used to have it where they built Olympic housing.”} However, a participant explained that the response and its nature would depend on the urgency and the severity of the crime being conducted. This discussion ended with a question about the type of crimes that predictive systems currently focus on or should focus on. Participants ask: \emph{“Are we thinking only of crime that can impact ...when we are thinking of crime, what is the definition of crime?” } As stated above, this question has repeatedly been asked in previous workshops. Deployment of police force was considered appropriate only for severe or violent crimes and therefore, participants wondered, what crimes were being predicted by the tool. 

Continuing the discussion on the type of crime the predictive systems focus on predicting, they recounted recent political crimes, that may be unusual, but very impactful. They asked: \emph{“Are they using it [predictive system] to figure out if next election we will have another charging of the capital? Where are the white supremacists? Are we looking for this?” } They considered how a predictive tool could have been useful in preventing historical crimes of a specific nature and asked if and how was this a possibility for preventing future crimes.

At numerous points during the workshop, participants drew on their own experience of space asking questions about specific neighborhoods. They wanted to know about seemingly wealthy neighborhoods that experience high crime rates. They asked \emph{“What happens in a neighborhood like Buckhead where the income level is quite high but there seem to a lot of crimes like gun shots.. You always hear about the Publix etc. How does that intersectionality work where there is high income but lots of crime?”} With this question, they wanted to know how the socio-political characteristics of a neighborhood may influence crime predictions and that means for the workings of the predictive tool. 

They also asked questions about the effect of changes in neighborhoods, such as social or infrastructural growth or decline, on crime predictions. Early on in the workshop, the group discussed the relation of food deserts, lack of economic opportunities and social resources, and access to healthcare and education, with crime. They observed how different areas in Atlanta are transforming and asked: \emph{ “I would be interested in how neighborhoods have changed over the years and how has that changed the data analytics... I wanna see how adding more grocery stores change an area.. it would be interesting to see what is the impact around Old Fourth Ward since they closed the hospital..” }. By tracking the changes in neighborhoods and their affect on crime, they hoped to identify methods to disincentivize crime by meeting basic human needs.  

Participants asked questions about what safety means for people in diverse neighborhoods. A participant laid out a detailed account where he compared two MARTA stations (MARTA is the public transit train system in Atlanta), East Point MARTA Station and College Park MARTA Station. He described how the College Park MARTA station has several private schools and has a large police force protecting students as they leave school. In contrast, East Point Marta Station, which is within 2 miles of College Park, has seen 2 murders in the last 4 months. Would this, they ask  \emph{“be an argument FOR predictive policing?” }Another participant adds nuance to this \emph{“With the tri-city moms there is a double-edged sword—I want to protect my kids but who am I protecting them from? Themselves or the police?” } They questioned how the presence of police may affect the feeling of safety in diverse neighborhoods. 

Above, I provide a report of the explanations contexts that emerged from meaningful questioning of AI systems.  I demonstrate how publics asked questions in large part due to their feelings about, knowledges of, or experiences with predictive domains (policing), predictive subjects (spaces), predictive backdrop (current events), and predictive tool (place-based predictive policing).  I discuss these relations in more detail in the following section. 