\section{Guiding future research: Questions to consider}

This dissertation calls for the need to \textit{bring partial explanations together, slowly but continuously, to diverse audiences especially those most at risk, in service of their diverse goals}. To that end, below I provide a list of questions that the XAI community can ask as they design for explaining civic AI systems to diverse publics. \\

\textbf{Situated:}
\begin{itemize}
    \item Who is aware of and has access to an explaining \textbf{site}? How can awareness and accessibility be promoted or channeled towards those most at risk of AI harms?
    \item What affordances does an explaining \textbf{medium} offer? Whose situated knowledges and learning needs are privileged or discounted and how? 
    \item How do explaining \textbf{interactions} mediate the process of knowing to meet diverse, situated, and long-term critical thinking needs? How do they support publics in drawing on their own expertise to effectively question or explain predictive systems?
    \end{itemize}
    
\textbf{Systemic: }

\begin{itemize}
    \item How does the explaining \textbf{site} allow for the space, time, and resources needed for systemic exploration of predictive systems? Who funds and supports these sites?
    \item What tools and languages does the explaining \textbf{medium} provide diverse publics to aid them in identifying, communicating, comparing, and organizing their partial explanations?
    \item How do the \textbf{interactions} allow diverse publics to identify components of AI systems they interact with, are knowledgeable of, or seek to learn about to promote systemic explaining and knowing?
\end{itemize}


\textbf{Slow and partial: }

\begin{itemize}
    \item How does the explaining \textbf{site} support long-lasting interactions for continued explaining and knowing? 
    \item How do explaining \textbf{media} allow gathering, organizing, and continuing slow explaining and knowing?
    \item How does the explaining \textbf{interaction} they support allow partial, but thoughtful creation and consumption of explanations? Does it account for changes in public perceptions and knowledges with time?
\end{itemize}

\textbf{Actionable:}

\begin{itemize}
    \item What connections and networks does the \textbf{site} support that are in service of the defined explaining \textbf{goals}?
    \item How do the explaining \textbf{media} along with the \textbf{interactions} allow publics to know and explain in ways that helps them formulate, communicate, and initiate necessary local actions for themselves and their communities?

\end{itemize}


